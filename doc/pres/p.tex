\documentclass[t]{beamer}
\usepackage[utf8]{inputenc}  % to be able to type unicode text directly
%\usepackage[french]{babel}   % french typographical conventions
\usepackage{inconsolata}     % for a nicer (e.g. non-courier) tt family font
\usepackage{amsthm,amsmath}  % fancier mathematics
\usepackage{mathabx}         % even fancier mathematics
\usepackage{array}           % to fine-tune tabular spacing
\usepackage{bbm}             % for blackboard 1
\usepackage{textpos}         % absolute positioning
\usepackage{hyperref,url}    % links and urls

\usepackage{graphicx}        % to include images
%\usepackage{animate}         % to include animated images
%\usepackage{bbding}          % for Checkmark and XSolidBrush
\usepackage[outputdir=PDFLATEXFILTERD]{minted}          % for code insets
\usepackage{soul}            % for colored strikethrough

\colorlet{darkgreen}{black!50!green}  % used for page numbers
\definecolor{term}{rgb}{.9,.9,.9}     % used for code insets

\setlength{\parindent}{0em}
\setlength{\parskip}{1em}


% coco's macros
\newcommand{\1}{\textbf{1}}
\def\R{\textbf{R}}
\def\C{\textbf{C}}
\def\N{\textbf{N}}
\def\T{\textbf{T}}
\def\F{\mathcal{F}}
\def\x{\textbf{x}}
\def\y{\textbf{y}}
\def\b{\textbf{b}}
\def\u{\mathbf{u}}
\def\Z{\textbf{Z}}
\def\d{\mathrm{d}}
\DeclareMathOperator*{\argmin}{arg\,min}
\DeclareMathOperator*{\argmax}{arg\,max}
\newcommand{\reference}[1] {{\scriptsize \color{gray}  #1 }}
\newcommand{\referencep}[1] {{\tiny \color{gray}  #1 }}
\newcommand{\unit}[1] {{\tiny \color{gray}  #1 }}

% \parens{x}      ->  (x)
% \pairing{x}{y}  ->  <x,y>
\newcommand{\parens}[1]{\left(#1\right)} % (x)
\newcommand{\pairing}[2]{\left\langle #1,\,#2\right\rangle} % <x,y>

% \abs{x}         ->    |x|
% \Abs{x}         ->   ||x||
% \ABS{x}         ->  |||x|||
\newcommand{\abs}[1]{\left|#1\right|}
\newcommand{\Abs}[1]{\left\|#1\right\|}
\newcommand{\ABS}[1]{{\left\vert\kern-0.25ex\left\vert\kern-0.25ex\left\vert #1 \right\vert\kern-0.25ex\right\vert\kern-0.25ex\right\vert}}

% \Sh -> dirac comb symbol shah
\DeclareFontFamily{U}{wncy}{}
\DeclareFontShape{U}{wncy}{m}{n}{<->wncyr10}{}
\DeclareSymbolFont{mcy}{U}{wncy}{m}{n}
\DeclareMathSymbol{\Sh}{\mathord}{mcy}{"58}

% really wide hat
\usepackage{scalerel,stackengine}
\stackMath
\newcommand\reallywidehat[1]{%
\savestack{\tmpbox}{\stretchto{%
  \scaleto{%
    \scalerel*[\widthof{\ensuremath{#1}}]{\kern-.6pt\bigwedge\kern-.6pt}%
    {\rule[-\textheight/2]{1ex}{\textheight}}%WIDTH-LIMITED BIG WEDGE
  }{\textheight}% 
}{0.5ex}}%
\stackon[1pt]{#1}{\tmpbox}%
}

% disable spacing around verbatim
\usepackage{etoolbox}
\makeatletter\preto{\@verbatim}{\topsep=0pt \partopsep=0pt }\makeatother

% disable headings, set slide numbers in green
\mode<all>\setbeamertemplate{navigation symbols}{}
\defbeamertemplate*{footline}{pagecount}{\leavevmode\hfill\color{darkgreen}
   \insertframenumber{} / \inserttotalframenumber\hspace*{2ex}\vskip0pt}

%%% select red color for strikethrough
\makeatletter
\newcommand\SoulColor{%
  \let\set@color\beamerorig@set@color
  \let\reset@color\beamerorig@reset@color}
\makeatother
\newcommand<>{\St}[1]{\only#2{\SoulColor\st{#1}}}
\setstcolor{red}

% make everything monospace
\renewcommand*\familydefault{\ttdefault}

\begin{document}

\begin{frame}[plain,fragile]
\LARGE
\begin{verbatim}




    ipol.sh  and  ipol.py





EML
GTTI 9--12--2020
\end{verbatim}
\end{frame}


\begin{frame}
IPOL.PY AND IPOL.SH\\
===================

{\bf What:}\\
{\color{blue}Natural} shell and python interfaces for many IPOL algorithms.

{\bf How:}\\
A single file {\color{blue}ipol.py} that can be imported from Python or run
from the command line.

{\bf Why:}\\
\pause
{\St{To troll Miguel.}}$ $
For various serious reasons.

\vfill
\pause
{\bf Status:}\\
- {\color{blue}github.com/mnhrdt/clipol}\\
- Works on Linux, BSDs and macOS Catalina, no Windows\\
- Missing support for octave/torch/tensorflow\\
- Must add ``all'' IPOL algorithms (only 10 yet)
\end{frame}


% THE WHAT
\begin{frame}%

\vfill
\begin{center}
\Huge
The \bf What
\end{center}
\vfill
\small
{$\,$ {\bf Natural interfaces} to IPOL algorithms in shell and Python}
\end{frame}

% what is a ``natural'' interface for LSD in shell and python
% (natural frontend)
\begin{frame}[fragile]
NATURAL PYTHON AND SHELL INTERFACES FOR LSD\\
===========================================

\vfill
{\bf ``Natural''} way to run LSD inside Python:
\begin{minted}[frame=single]{python}
import ipol       # the proposed interface to IPOL
x = ...           # create an image as an ndarray
y = ipol.lsd(x)   # y is now an array of size Nx7
\end{minted}


\vfill
{\bf ``Natural''} way to run LSD inside the shell:
\begin{minted}[frame=single]{bash}
ipol lsd image.png segments.txt
\end{minted}

\vfill
\end{frame}

% ease to concatenate demos: e.g. ponomarenko+denoising
\begin{frame}[fragile]
CONCATENATION OF IPOL ALGORITHMS\\
================================

{\bf Goal:}
Estimate the noise of an image using Ponomarenko's method, then denoise it
using BM3D.

\vfill
In Python:
\begin{minted}[frame=single]{python}
import ipol, iio
x = iio.read("barbara.png")
s = ipol.ponomarenko(x)
y = ipol.bm3d(x, s)
iio.write("barbara_denoised.png")
\end{minted}

\vfill
In the shell:
\begin{minted}[frame=single]{bash}
S=`ipol ponomarenko barbara.png`
ipol bm3d --sigma=$S barbara.png barbara_denoised.png
\end{minted}

\end{frame}

% python help (with completion)
\begin{frame}
PYTHON HELP VIA COMPLETION\\
==========================

\end{frame}

% shell help (with completion)
\begin{frame}
SHELL HELP OPTION\\
=================

\end{frame}

% frontend design criteria
\begin{frame}
EXISTING ALGORITHMS\\
===================

Some IPOL algorithms have already been adapted to this system:
{\color{blue}
ldf, tvl1flow, phsflow, roflow, rdpoflow, ace, bm3d, sift, asift,
goldstein-fattal, canny-devernay, retinex-poisson, nlm, dct-denoise
}

The adaptation consists in reading and understanding the README file of the
code, and call it from the command line.  Equivalent to writing the
{\color{blue} run.sh} file in the demo system.  Estimated time per
algorithm:~{\bf 15min}.
\end{frame}

% frontend design criteria
\begin{frame}
FRONTEND DESIGN CRITERIA\\
========================

1. Each algorithm has a
{\color{blue}unique},
{\color{blue}meaningful
%well-chosen
name}.

2. The interface is as natural as possible, and selects {\color{blue}reasonable
defaults} for missing parameters.

3. Raw access to the original executable is available via the option
{\color{blue}--raw}.

4. Transparent conversion to the image format chosen by each program.  The
default file format is {\color{blue}.npy}.
\end{frame}

% ``natural'' bakend: idl
\begin{frame}[fragile]
BACKEND: IDL FILES (INSPIRED BY DOCKERFILES)\\
============================================

Complete {\bf IDL} file for LSD:

\begin{minted}[frame=single,fontsize=\small]{docker}
NAME lsd
TITLE Line Segment Detector
SRC http://www.ipol.im/pub/art/2012/gjmr-lsd/lsd_1.6.zip

INPUT in image:pgm
OUTPUT out text

BUILD make
BUILD cp lsd $BIN

RUN lsd $in $out
\end{minted}
\end{frame}

\begin{frame}[fragile]
BACKEND: IDL FILES (INSPIRED BY DOCKERFILES)\\
============================================

Complete {\bf IDL} file for LSD:

\begin{minted}[frame=single,fontsize=\small]{docker}
NAME lsd  # global identifier
TITLE Line Segment Detector
SRC http://www.ipol.im/pub/art/2012/gjmr-lsd/lsd_1.6.zip

INPUT image:pgm in  # the input image is converted to pgm
OUTPUT text out

BUILD make          # shell instructions to build the program
BUILD cp lsd $BIN   # program files are copied to $BIN

RUN lsd $in $out    # $BIN is in the path, you can run lsd
\end{minted}
Comments are allowed!
\end{frame}

% a more complex idl with per-uname differences
\begin{frame}[fragile]
A MORE COMPLEX IDL FILE\\
=======================

\begin{minted}[frame=single,fontsize=\footnotesize]{docker}
a
\end{minted}
\end{frame}


% list of current IDL files
\begin{frame}
EXISTING IDL FILES\\
==================

\end{frame}




% THE HOW
\begin{frame}%

\vfill
\begin{center}
\Huge
The \bf How
\end{center}
\vfill
\small
\centering
{A single file {\bf ipol.py}}
\end{frame}


% how does it work, what does it do internally?
\begin{frame}
INNER WORKINGS OF ipol.py\\
=========================

\end{frame}

% the inner working is an implementation detail
% comment: enforcing stricter software guidelines would make an easier, more
% efficient integration trivial
%\begin{frame}
%IPOL.PY AND IPOL.SH\\
%===================
%
%\end{frame}


% THE WHY
\begin{frame}%

\vfill
\begin{center}
\Huge
The \bf Why
\end{center}
\vfill
\small
\centering
{Various serious reasons}
\end{frame}

\begin{frame}
$\!\,${\bf WHY:} SUMMARY OF REASONS\\
=======================

1. Proof-based reproducibility\\
(as opposed to trust-based for ipol demos)

2. Personal usage: avoid re-reading the README file every time I want to run an
ipol code.

3. Personal challenge: write a sane Python program

4. Towards a ``corpus'' of algorithms like in megawave

\end{frame}

% trust-based reproducibility in current ipol demos
% goal: proof-based reproducibility
\begin{frame}
IPOL.PY AND IPOL.SH\\
===================

\end{frame}


% in more practical terms: avoid reading the ''README'' for each new ipol code
% we want to try.  The IDL formalizes (and thus replaces) the README file
\begin{frame}
IPOL.PY AND IPOL.SH\\
===================

\end{frame}


% personally, a proof of sane python coding style
% - indentation with tabs
% - all variables are single letters
% - no global imports
% - no bullshit like json, pandas, and the like
\begin{frame}
IPOL.PY AND IPOL.SH\\
===================

\end{frame}


% build a corpus of algorithm à la megawave that are runnable and comparable
% together.  Goal: never get to ``de-ipolisize'' a code in the future.
\begin{frame}
IPOL.PY AND IPOL.SH\\
===================

\end{frame}


% design phylosopy (from more important to less important)
% 1. Super-simple shell interface
% 2. Super-simple Python interface
% 3. Super-simple IDL files (goal: write a new IDL file in less than 15min)
% 4. Minimum number of Python dependencies
% 5. Super-simple Python implementation
% 6. Portability to as many IPOL algorithms as possible
% 7. Portability to as many systems as possible
% 8. Efficient runtime
\begin{frame}
IPOL.PY AND IPOL.SH\\
===================

\end{frame}

% status and future work
\begin{frame}
STATUS AND FUTURE WORK\\
======================

\end{frame}

% support promise: 1 idl file = 1 pint of beer (of my choice)
\begin{frame}
IPOL.PY AND IPOL.SH\\
===================

\end{frame}

\end{document}


% vim:sw=2 ts=2 spell spelllang=fr:
